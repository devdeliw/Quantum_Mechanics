We will now implement the prime directive for several different potentials, the
first of which is the infinite square well. 


\section{The Infinite Square Well}

A particle of mass $m$ is confined to a region of width $a$, $-a/2 < x < a/2$,
by the potential 

 \[
V(x) = \begin{cases}
  0 &\qquad |x| < a/2 \\
  \infty &\qquad \text{otherwise} 
\end{cases} 
\] \vspace{3px}

\begin{figure}[H]
  \centering
    \includegraphics[width = 6.8cm]{infinitesquarewell.pdf}
    \caption{The Infinite Square Well}
\end{figure}

Aligning ourselves with the prime directive, we look for solutions of Schr\"odinger equation of the
form 

\[
  \Psi(x, t) = \sum_i \psi(x)e^{-i E_i t/\hbar} \qquad \left[
  -\frac{\hbar^2}{2m} \frac{d^2 }{d x^2} + V(x) \right] \psi_i(x) = E_i
  \psi_i(x) 
\] \vspace{3px}

The solution outside the well is $\psi_i(x) = 0$. This will be demonstrated
later in the chapter by solving the finite square well, then taking the
infinite well limit. But intuitively it makes sense, since as the outside
potential is infinite, the wave function can not penetrate past the boundary at
all. \\

Inside the well we solve 

\[
-\frac{\hbar^2}{2m} \frac{d^2 }{d x^2} \psi_i(x) = E_i \psi_i(x) \quad
\Rightarrow \quad \frac{d^2 }{d x^2} \psi_i(x) = -k_i^2 \psi_i(x)
\] \vspace{3px}   


where $k_i = \frac{\sqrt{2mE_i}}{\hbar}$. This differential equation has the
general solution (verify) of 

\[
\psi_i(x) = A\sin k_ix + B\cos k_i x
\] \vspace{3px}

We now deal with the boundary conditions. The wave functions must vanish at
$|x| = a/2$, as it vanishes for all  $|x| > \frac{a}{2}$, and the wave function
must also be continuous. This requires 


\[
  \lim_{x \to \pm a} \psi_i(x) = 0
\] \vspace{3px}

Substituting in for $\psi_i(x)$, and evaluating the limit, we get

\begin{align}
&A\sin \frac{k_i a}{2} + B\cos \frac{k_i a}{2} = 0 \label{first}\\
&A\sin \left( -\frac{k_ia}{2} \right) + B\cos \left( -\frac{k_i a}{2} \right)
= -A \sin \frac{k_i a}{2} + B\cos \frac{k_i a}{2} = 0 \label{second} 
\end{align} \vspace{3px}

Adding and subtracting Equations \ref{first} and \ref{second},

\begin{align} \label{}
  &A\sin \frac{k_i a}{2} + B\cos \frac{k_i a}{2} + \left( - A \sin \frac{k_i
  a}{2} + B\cos \frac{k_i a}{2}\right)  \quad \Rightarrow \quad B\cos \frac{k_i
a}{2} = 0 \\
  &A\sin \frac{k_i a}{2} + B\cos \frac{k_i a}{2} - \left( - A \sin \frac{k_i
  a}{2} + B\cos \frac{k_i a}{2} \right) \quad \Rightarrow \quad A \sin
  \frac{k_i a}{2} = 0 
\end{align}\vspace{3px}

Therefore we yield two sets of solutions, 

\begin{align} \label{}
  \psi_n(x) = \begin{cases}
    B \cos k_n x &\quad k_n = \{ \frac{\pi}{a}, \frac{3\pi}{a}, \cdots \}
    = \frac{\pi n }{a}, \;  n = 1, 3, 5, \cdots \qquad \text{ even parity } \\
    A \sin k_n x &\quad k_n = \{ \frac{2\pi}{a}, \frac{4\pi}{a}, \cdots \}
    = \frac{\pi n}{a}, \;  n = 2, 4, 6, \cdots \qquad \text{odd parity} 
  \end{cases} 
\end{align}\vspace{3px}

We replace $i$ with $n$, the quantum number that ``labels" our solutions. The
even parity solutions for odd $n$ occur when the stationary state wave function is symmetric
about the origin. The odd parity solutions are antisymmetric. Now, knowing that 

\[
k_n = \frac{\sqrt{2mE_n}}{\hbar} \Rightarrow E_n = \frac{\hbar^2 k_n^2}{2m}
= \frac{\hbar^2 n^2 \pi^2}{2ma^2}
\] \vspace{3px}

We get the allowed energy eigenvalues that correspond to the stationary states
labeled with quantum number $n$. Now we move onto normalization to determine
the coefficients $A$ and $B$. Using Equation \ref{eq:normalization}, 

\begin{align} \label{}
  &\int_{-a/2}^{a/2} |\psi_{i, \text{even parity}} (x)|^2 \, dx = \int_{-a/2}^{a/2} B^2 \cos^2
  \frac{\pi n x}{a} \, dx = B^2 \frac{a}{2} = 1 \quad \Rightarrow \quad
  B = \sqrt{\frac{2}{a}} \\ 
  &\int_{-a/2}^{a/2} |\psi_{i, \text{ odd parity}} (x) |^2  \, dx
  = \int_{-a/2}^{a/2} A^2 \sin^2 \frac{\pi n x}{a} \, dx = A^2 \frac{a}{2}
  = 1 \quad \Rightarrow \quad A = \sqrt{\frac{2}{a}}
\end{align}\vspace{3px}

Therefore, our normalized stationary state solutions, \\

\begin{mainbox}{Infinite Square Well Stationary States}
  \[ \psi_n(x) = \begin{cases}
    & \sqrt{\frac{2}{a}}\cos \frac{\pi n x}{a}, \; n = 1,3,5, \cdots \quad
    \text{ even parity} \\ & \sqrt{\frac{2}{a}} \sin \frac{\pi n x}{a}, \;
    n = 2, 4, 6, \cdots \quad \text{ odd parity} 
  \end{cases}  \] \[ E_n = \frac{\hbar^2 k_n^2}{2m} = \frac{\hbar^2 n^2
\pi^2}{2ma^2} \]
\end{mainbox} \vspace{5px}

I plot the first four stationary state solutions along with their defined energies
in Figure \ref{fig:infsquarewell}.
\begin{figure}[!ht]
  \centering
    \includegraphics[width = 11cm]{infinitesquarewellsolutions.pdf}
    \caption{Infinite Square Well Stationary States}
    \label{fig:infsquarewell}
\end{figure}
\subsection{Various Comments about the Solutions}

\begin{itemize}
  \item[1.] There are an infinite number of allowed energy eigenvalues and
    eigenfunctions labeled by the discrete quantum number index $n$. This
    is a consequence of the boundary condition that $\psi(x)$ vanish at $|x|
    < a/2$, limiting solutions to integer and half-integer wavelengths. 
  \item[2.] The basis is orthonormal as  \[ \int_{-a/2}^{a/2} \psi_n'^*(x)
    \psi_n(x) \, dx = \delta_n'n \] can be readily verified. 
  \item[3.] The basis is complete for any function $\Psi(x)$ satisfying the
    boundary condition $\Psi(a/2) = \Psi(-a/2) = 0$ and defined on the interval
    $[-a/2, a/2]$. Any function satisfying these conditions can be expanded in
    this basis. Those of you familiar with Fourier Series may have noticed that
    the odd basis functions \[ \sqrt{\frac{2}{a}}\sin \left( \frac{\pi n x}{a}
      \right), \; n = 2, 4, 6, \cdots \; \text{ which can be written as
      } \; \sqrt{\frac{2}{a}}\sin \left( \frac{\pi n' x}{a/2} \right), \; n' = 1, 2,
    3, \cdots \] \vspace{3px} are the standard odd functions of a Fourier Series, while the
    even basis functions \[ \sqrt{\frac{2}{a}}\cos \frac{\pi n x}{a}, \; n = 1, 3,
      5, \cdots \text{ which can be written as } \; \sqrt{\frac{2}{a}}\cos \left(
    \frac{\pi (n' - \frac{1}{2})x}{a/2}\right), \; n' = 1, 2, 3, \cdots \] \vspace{5px} have
    been shifted in index, and the constant term  $(n' = 0)$ is absent. These
    modifications reflect restrictions imposed on the basis by our use of
    specific boundary conditions.  
  \item[4.] These wave functions have  $n-1$ interior zeroes -- coordinates at
    which the probability to find the trapped particle vanishes. 
  \item[5.] The eigenfunctions have alternating definite parity -- even or odd
    -- a consequence of the reflection symmetry of the potential.
\end{itemize}

\subsection{Wave Function Curvature}

Within the well interior the particle propagates as a free particle --
a particle in the absence of any confining potential. One can understand the
physics of our well solutions from the correlation between wave function
curvature and momentum. The momentum operator \textit{measures} curvature, and
our energy is quadratic in $p$. This can be made explicit by evaluating the
expectation value of $\langle p^2 \rangle$ between the stationary states

\[
  \langle \hat{p}^2 \rangle = \frac{\hbar^2 \pi^2 n^2}{a^2} \quad \text{ so
  that }  \quad \frac{1}{2m}\langle \hat{p}^2 \rangle = \langle \hat{H} \rangle 
\] \vspace{3px}
For a fixed number of nodes, doubling $a$ will half the curvature. Therefore,
energies depend \textit{inversely} on $a^2$. Conversely doubling the number of
nodes doubles the curvature. Consequently energies scale as $n^2$. 

The square well is exceptional in that it confines all wave functions in the
same way. This leads to the steep $n^2$ dependence of energy eigenvalues. In
a finite well -- a future exercise -- where the boundaries are not infinitely
strong -- the wave functions at higher excitation energies \textit{penetrate}
into the classically-forbidden region of the potential, reducing the curvature
and thus producing energies that increase less steeply than $n^2$. This is
known as \textit{\textbf{quantum tunneling}}. In the harmonic oscillator --
another future exercise -- the widening $r^2$ potential leads to a spectrum
that is evenly spaced, with eigenvalues rising with $n$. 

\subsection{Example Problem -- The Prime Directive}

We calculated the stationary states for the infinite square well above. But
what is the full, time-dependent normalized wave function? How do you find the
$c_i's$ in Equation \ref{primedirec}?. \\

To determine these, we have to have an additional boundary condition, namely
the wave function value at some time  $t$, call it $t=0$. 

\paragraph{Example} A particle in an infinite square well has the initial wave
function shown in Figure \ref{fig:prob}.

\begin{align} \label{theproblem}
  \Psi(x, 0) = \begin{cases}
    Ax(a-x) &\qquad 0 \leq x \leq a \\ 0 &\qquad x > 0
  \end{cases} 
\end{align}\vspace{3px}

for some constant $A$. Find $\Psi(x, t)$. 

\begin{figure}[htp]
  \centering
    \includegraphics[width = 12cm]{infinitesquarewell_problem.pdf}
    \caption{$\Psi(x, 0)$}
    \label{fig:prob}
\end{figure}

First we must determine $A$ using the normalization condition. If $\Psi(x,
0)$ is normalized, $\Psi(x, t)$ will stay normalized. 

\begin{align} \label{}
  \int_{0}^{a} |\Psi(x, 0)|^2 \, dx = |A|^2\int_{0}^{a} x^2(a-x)^2 \, dx
  &= |A|^2 \int_{0}^{a} (a^2x^2 - 2ax^3 + x^4) \, dx \\ &= |A|^2\left( a^2
  \frac{x^3}{3} - 2a \frac{x^4}{4} + \frac{x^5}{5} \right) \Bigg|_0^a = |A|^2
  \left( \frac{a^5}{3} - \frac{2a^5}{4} + \frac{a^5}{5} \right) = |A|^2
  \frac{a^5}{30} = 1 \\ &\Rightarrow |A| = \sqrt{\frac{30}{a^5}} 
\end{align}\vspace{3px}

Now we can determine the coefficients $c_n$ using Equation \ref{primedirec}. 

\begin{align*}
  \Psi(x, 0) &= \sum_{n=1}^{\infty} c_n \psi_n(x) \\
  c_n &= \int \psi_n(x) \Psi(x, 0) \, dx \\
      &= \sqrt{\frac{2}{a}}\int_{0}^{a} \sin\left( \frac{n\pi}{a}x \right)
      \sqrt{\frac{30}{a^5}}x(a-x)\,dx \\
      &= \frac{2\sqrt{15}}{a^3}\left[a\int_{0}^{a} x\sin\left( \frac{n\pi}{a}x
      \right) \, dx - \int_{0}^{a} x^2 \sin \left( \frac{n\pi}{a}x \right) \,
dx\right] \\ &=\frac{2\sqrt{15}}{a^3} \left[ a \left[ \left( \frac{a}{n\pi}
    \right) ^2 \sin \left( \frac{n\pi}{a}x \right) - \frac{ax}{n\pi}\cos \left(
    \frac{n\pi}{a}x\right) \right]_0^a - \left[ 2 \left( \frac{a}{n\pi} \right)
  ^2 x \sin \left( \frac{n\pi}{a}x \right) - \frac{(n\pi x a)^2 - 2}{(n\pi
a)^3}\cos \left( \frac{n\pi}{a}x \right)  \right]_0^a \right]     \\
             &= \frac{2\sqrt{15}}{a^3} \left[ -\frac{a^3}{n\pi}\cos(n\pi)
               + a^3\frac{(n\pi)^2 - 2}{(n\pi)^3}\cos(n\pi) + a^3
             \frac{2}{(n\pi)^3}\cos(0)\right] \\
             &= \frac{4\sqrt{15}}{(n\pi)^3}[\cos(0) - \cos(n\pi)]\\
             &=\begin{cases}
             0 &\qquad \text{if $n$ even} \\
               \frac{8\sqrt{15}}{(n\pi)^3} &\qquad \text{if $n$ odd}
             \end{cases}
\end{align*}

So now we can form our time-dependent wave function using the prime directive by putting everything
together. We get 

\begin{align}
  \Psi(x, t) &= \sum_{n=1}^\infty c_n \psi_n(x) e^{-iE_n t / \hbar} \\ 
  \Psi(x, t) &= \sqrt{\frac{30}{a}} \left( \frac{2}{\pi} \right) ^3 \sum_{n
= 1,3,5,\cdots} \frac{1}{n^3}\sin \left( \frac{n\pi x}{a} \right) e^{-in^2\pi^2
\hbar t / 2ma^2} \label{final}
\end{align} \vspace{3px}

Notice the solution only contains the $\sin \left( \frac{n\pi x}{a} \right)$
component. This is because the square well in this problem is
defined from $0 < x < a$ and not reflection symmetric around the origin. The final
solution in Equation \ref{final} describes how the wave function changes as
a function of time, provided the particle is in a square well. \\

Now let us study this wave function, and its implications on observables. We
first determine what the expectation value of the energy is. 

\begin{align}
  \langle E \rangle &= \int \Psi(x, t)^* \hat{H} \Psi(x, t) \, dx \\ 
                    &= \int \Psi(x, t)^* E_n \Psi(x, t)\, dx
\end{align}

Performing the calculation, one will find 

\[
\langle E \rangle = \sum_n |c_n|^2 E_n
\] \vspace{3px}

So we can think of $|c_n|^2$ as the probability to measure the energy
eigenvalue $E_n$. Therefore, 

\[
\sum_m |c_n|^2 = 1
\] \vspace{3px}

The $c_n^2$ can also be thought of as telling us the ``amount" of $\Psi_n$ that
is in the \textit{total} wave function. In the example above, we can see that
the initial wave function closely resembles $\Psi_1$. If we look at $c_1$, 

\[
|c_1|^2 = \left( \frac{8\sqrt{15}}{\pi^3} \right) ^2 = 0.998555... 
\] \vspace{3px}

we see that it is very close to $1$, indicating that the $n=1$ state dominates.
\\

The Infinite Square Well is the classic introductory quantum mechanics problem.
Before moving on, ensure you understand every derivation in this section
\textit{completely.} 

\section{The Harmonic Oscillator}

This next section is \textit{very} important. In nature, ``friggin' everything is
a harmonic oscillator" -- Reddit commenter. The harmonic oscillator stationary-state basis is
arguably the most versatile and important in physics. Every field in physics
includes key problems that require one to understand small-amplitude behavior
that maps onto the harmonic oscillator. 

The \textit{quantum} 1D harmonic oscillator is to solve the Schr\"odinger
equation for the potential shown in Figure \ref{fig:harmonicoscillator}

\begin{align} \label{HOpotential}
  V(x) = \frac{1}{2}m\omega^2 x^2
\end{align}\vspace{3px}

\begin{figure}[!ht]
  \centering
    \includegraphics[width = 8cm]{harmonicoscillator.pdf}
    \caption{Quantum Harmonic Oscillator Potential}
    \label{fig:harmonicoscillator}
\end{figure}

Therefore, by the prime directive, the time-independent Schr\"odinger equation
reads

\begin{align} \label{HOschr}
  -\frac{h^2}{2m} \frac{d^2 \psi}{d x^2} + \frac{1}{2}m\omega^2 x^2 \psi
  = E\psi
\end{align}\vspace{3px}

There are two methods to solve this problem. The first is the ``brute force"
attempt to solve the differential equation using power series. The second,
according to Griffiths, is a ``diabolically clever" technique using
\textit{ladder operators}. We will start with the ladder operator technique. 

\subsection{Ladder Operator Algebraic Technique}

We rewrite Equation \ref{HOschr} utilizing the momentum operator: 

\begin{align} \label{}
  \frac{1}{2m}\left[ \hat{p}^2 + (m\omega w)^2 \right] \psi = E\psi
\end{align}\vspace{3px}

where $\hat{p}\equiv -i\hbar \frac{d}{dx}$ is the momentum operator. We define
the ladder operators as the following: 

\begin{align} \label{}
  \hat{a}_\pm \equiv \frac{1}{\sqrt{2\hbar m\omega}}(\mp i \hat{p} + m\omega x)
\end{align}\vspace{3px}

You might be wondering why the following operators are known as \textit{ladder}
operators -- we'll get to that in a bit, but for now, let us determine the
product $\hat{a}_- \hat{a}_+$. 

\begin{align} \label{}
  \hat{a}_- \hat{a}_+ &= \frac{1}{2\hbar m \omega} (i \hat{p} + m\omega x)(-i
  \hat{p} + m\omega x) \\ &= \frac{1}{2\hbar m \omega} \left[ \hat{p}^2
  + (m\omega x)^2 - im\omega (x \hat{p} - \hat{p} x) \right]
\end{align}\vspace{3px} 

Note above, we do not combine the terms $(m\omega x)(-i \hat{p})$ and $(i
\hat{p})(m\omega x)$. This is because in we are dealing with operators.
Operators do not, in general, \textbf{commute} ($x \hat{p} \neq \hat{p}x$ ). So
we have to separate the two. As a result, there is an extra term involving ( $x
\hat{p} - \hat{p} x$ ). We call this the \textbf{commutator} of $x$ and
$\hat{p}$. In general, the commutator of operators $\hat{A}$ and $\hat{B}$ is 

\begin{align} \label{commutator}
  [\hat{A}, \hat{B}] \equiv \hat{A}\hat{B} - \hat{B}\hat{A}
\end{align}\vspace{3px}

Using this notation, 

\begin{align} \label{a-a+}
  \hat{a}_- \hat{a}_+ = \frac{1}{2\hbar m \omega}\left[ \hat{p}^2 + (m\omega
  x)^2 \right] - \frac{i}{2\hbar}[x, \hat{p}]
\end{align}\vspace{3px}

And so we need to figure out the commutator of $x$ and $\hat{p}$. To do this,
we employ an arbitrary ``test`` function $f$ to see what the \textit{effect} of
the commutator is. At the end we can then throw away $f$ to determine the value
of the commutator. We have 

\begin{align} \label{}
  [x, \hat{p}] f(x) &= \left[ x(-i\hbar) \frac{d }{d x} f(x) - (-i\hbar) \frac{d
  }{d x} (xf) \right] \\ &= -i\hbar \left( x \frac{d f}{d x} - x \frac{d f}{d x}
- f\right) \\ &= i\hbar f(x)
\end{align}\vspace{3px}   

Therefore, $[x, \hat{p}] = i\hbar$. This formula is known as the \textbf{canonical
commutation relation}. With this, Equation \ref{a-a+} becomes 

\begin{align} \label{}
  \hat{a}_- \hat{a}_+ &= \frac{1}{\hbar \omega} \hat{H} + \frac{1}{2}  \\
                      &\Rightarrow \hat{H} = \hbar \omega \left( \hat{a}_-
                      \hat{a}_+ - \frac{1}{2} \right)      
\end{align}\vspace{3px}

Notice that the ordering of $\hat{a}_+$ and $\hat{a}_-$ is important here; the
same argument with $\hat{a}_+$ on the left, yields 

\begin{align} \label{hami}
  \hat{a}_+ \hat{a}_- = \frac{1}{\hbar \omega} \hat{H} - \frac{1}{2}
\end{align}\vspace{3px}   

Therefore, another expression for the Hamiltonian can be derived by rearranging
Equation \ref{hami}. 

\begin{align} \label{}
  \hat{H} = \hbar \omega \left(\hat{a}_+ \hat{a}_- + \frac{1}{2}\right)
\end{align}\vspace{3px}


Then, in terms of $\hat{a}_\pm$, the Schr\"odinger equation for the harmonic
oscillator can be written as 

\begin{align} \label{intermsof}
  \hbar\omega \left( \hat{a}_\pm \hat{a}_\mp \pm \frac{1}{2} \right) \psi = E\psi 
\end{align}\vspace{3px}

Now comes the pinnacle of the ladder operator method -- the reason it is
``diabolically clever." \\

\begin{mainbox}{Energy Eigenvalues of Ladder Operator}
If $\psi$ satisfies the Schr\"odinger equation with energy $E$
(that is, $\hat{H}\psi = E\psi$), then $a_+ \psi$ satisfies the Schr\"odinger
equation with energy $(E+\hbar \omega)$ (that is, $\hat{H}(\hat{a}_+ \psi)
= (E+\hbar \omega)(\hat{a}_+\psi)$.
\end{mainbox} \vspace{3px}

Because this is so important and may seem like it came out of the blue,
I provide a proof. 

\begin{align}
  \hat{H}(\hat{a}_+\psi) &= \hbar\omega\left(\hat{a}_+\hat{a}_-
  + \frac{1}{2}\right)(\hat{a}_+\psi) = \hbar\omega \left(
\hat{a}_+\hat{a}_-\hat{a}_+ + \frac{1}{2}\hat{a}_+ \right) \psi \\ 
                         &= \hbar \omega \hat{a}_+ \left( \hat{a}_-\hat{a}_+
                         + \frac{1}{2} \right) \psi = \hat{a}_+ \left[
                         \hbar\omega \left( \hat{a}_+ \hat{a}_-
                       + 1 + \frac{1}{2} \right) \psi\right] \\
      &= \hat{a}_+(\hat{H} + \hbar \omega) \psi = \hat{a}_+(E+\hbar\omega)\psi
      = (E +\hbar\omega)(\hat{a}_+\psi). \qquad \text{QED}.
  \end{align} \vspace{3px}

Note that in the second line I replaced $\hat{a}_-\hat{a}_+$ by
$(\hat{a}_+\hat{a}_- + 1)$. I can do this because (verify) $[\hat{a}_-,
\hat{a}_+] = 1$. By the same procedure, $\hat{a}_-\psi$ is a solution as well,
with energy eigenvalue $(E - \hbar\omega)$. 

\begin{align} \label{}
  \hat{H}(\hat{a}_-\psi) &= \hbar\omega \left( \hat{a}_-\hat{a}_+ - \frac{1}{2}
  \right) (\hat{a}_-\psi) = \hbar\omega \hat{a}_- \left( \hat{a}_+ \hat{a}_-
  - \frac{1}{2} \right) \psi \\ &= \hat{a}_-\left[ \hbar\omega \left(
\hat{a}_-\hat{a}_+ - 1 -\frac{1}{2} \right) \psi\right] = \hat{a}_- \left(
  \hat{H} - \hbar\omega \right) \psi = \hat{a}_-(E-\hbar\omega)\psi \\ &=
  (E-\hbar\omega)(\hat{a}_-\psi).   
\end{align}\vspace{3px}


This is why we call $\hat{a}_\pm$ ladder operators. If we could just find
\textit{one} solution, we can use $\hat{a}_\pm$ to ``climb up and down" in
energy, getting \textit{all} the possible energy eigenvalues. Hence, we call
$\hat{a}_+$ the \textbf{raising operator} and $\hat{a}_-$ the \textbf{lowering
operator}. 

Let us try to find the \textit{lowest rung}, $\psi_0$, such that
$\hat{a}_-\psi_0 = 0$. Therefore, 

\begin{align} \label{}
  \frac{1}{\sqrt{2\hbar m \omega}} \left( \hbar \frac{d }{d x} + m\omega
  x \right) \psi_0 = 0 
\end{align}\vspace{3px}

Rearranging gives 

\begin{align} \label{}
  \frac{d \psi_0}{d x} = -\frac{m\omega}{\hbar}x\psi_0
\end{align}\vspace{3px}

Implementing separation of variables, 

\begin{align} \label{}
  \int \frac{1}{\psi_0} \, d\psi_0 = -\frac{m\omega}{\hbar} \int x \, dx \quad
  \Rightarrow \quad \ln \psi_0 = -\frac{m\omega}{2\hbar} x^2 + \text{const.} 
\end{align}\vspace{3px}

Hence, 

\begin{align} \label{}
  \psi_0(x) = A e^{-\frac{m\omega}{2\hbar}x^2}
\end{align}\vspace{3px}

Normalizing, 

\begin{align} \label{}
  1 = |A|^2 \int_{-\infty}^{\infty} e^{-m\omega x^2 / \hbar} \, dx
    = |A|\sqrt{\frac{\pi \hbar}{m\omega}} \quad \Rightarrow \quad A^2
    = \sqrt{m\omega / \pi \hbar}
\end{align}\vspace{3px}

Therefore, we get the final ``lowest-rung" stationary state 

\begin{align} \label{lowestrung}
  \psi_0(x) = \left( \frac{m\omega}{\pi \hbar} \right) ^{1/4}
  e^{-\frac{m\omega}{2\hbar}x^2} 
\end{align}\vspace{3px}

To determine the energy of this state we plug it into the Schr\"odinger
Equation \ref{intermsof} and exploit the fact that $\hat{a}_-\psi_0 = 0$. ]


\begin{align} \label{}
  \hbar\omega\left(\hat{a}_+\hat{a}_- + \frac{1}{2}\right)\psi_0 &= E_0\psi_0\\
  \hbar\omega\left(\hat{a}_+(\hat{a}_-\psi_0) + \frac{1}{2}\psi_0\right) &=
  E_0\psi_0 \\ \hbar\omega \left( 0  + \frac{1}{2}\psi_0\right)  = E_0\psi_0
\end{align}\vspace{3px}

Therefore, $E_0 = \frac{1}{2}\hbar\omega$. We have secured the \textbf{ground
state} of the quantum harmonic oscillator. We now can just apply $\hat{a}_+$
repeatedly to generate the excited states, increasing the energy by
$\hbar\omega$ each step as we go along. Therefore, every stationary state,
labeled by a quantum number $n$ can be defined as follows:

\begin{align} \label{harmenergy}
   \psi_n(x) = A_n(\hat{a}_+)^n \psi_0(x), \qquad E_n = \left(
  n+\frac{1}{2} \right) \hbar\omega 
\end{align}\vspace{3px}

where $A_n$ is the normalization constant. You can get $A_n$ algebraically,
however, and the proof is laid out in Griffiths. It turns out that $A_n
= \frac{1}{\sqrt{n!}}$. Thus, 

\begin{align} \label{laddersolution}
  \psi_n &= \frac{1}{\sqrt{n!}}(\hat{a}_+)^n\psi_0 
\end{align}
 \[ \boxed{
  \psi_n = \frac{1}{\sqrt{n!}} (\hat{a}_+)^n \left( \frac{m\omega}{\pi\hbar}
\right)^{\frac{1}{4}} e^{-\frac{m\omega}{2\hbar}x^2} }
  \] \vspace{3px}

\subsection{Example -- Expectation Value of V(x)}

\paragraph{Example} Find the expectation value of the potential energy in the
$n $th stationary state of the harmonic oscillator. 

\[
\langle V \rangle = \langle \frac{1}{2}m\omega^2 x^2\rangle
= \frac{1}{2}m\omega^2 \int_{-\infty}^{\infty} \psi_n^* x^2 \psi_n \, dx
\] \vspace{3px}

Expressing $x$ and $\hat{p}$ in terms of the raising and lowering operators, 

\[ \boxed{ x = \sqrt{\frac{\hbar}{2m\omega}} (\hat{a}_+ + \hat{a}_-); \qquad
\hat{p} = i\frac{\hbar m \omega}{2} (\hat{a}_+ - \hat{a}_-) } \]\vspace{3px}

Therefore, 

\[
x^2 = \frac{\hbar}{2m\omega} \left[ (\hat{a}_+^2 + \hat{a}_+\hat{a}_-
+ \hat{a}_-\hat{a}_+ + \hat{a}_-^2 \right] 
\] \vspace{3px}

Hence, 

\[
\langle V \rangle = \frac{\hbar\omega}{4}\int_{-\infty}^{\infty}  \psi_n^*
\left[ (\hat{a}_+^2 + (\hat{a}_+\hat{a}_-) + (\hat{a}_-\hat{a}_+)
+ \hat{a}_-^2\right] \psi_n \, dx
\] \vspace{3px}

But $\hat{a}_+^2\psi_n$ is (apart from normalization), just $\psi_{n+2}$, which
is orthogonal to $\psi_n$ and the same goes for $\hat{a}_-^2\psi_n \sim
\psi_{n-2}$. So those terms drop out, and we are left with 

\[
\langle V \rangle = \frac{\hbar\omega}{4}\int_{-\infty}^{\infty}  \psi_n^*
\left[ (\hat{a}_+\hat{a}_-) + (\hat{a}_-\hat{a}_+) \right] \psi_n \, dx
\] \vspace{3px}

Now, without proof (but check Griffiths) I will use the following relations: 

\[
\hat{a}_+\hat{a}_- \psi_n = n\psi_n, \qquad \hat{a}_-\hat{a}_+\psi_n
= (n+1)\psi_n
\] \vspace{3px}

Therefore, we get the final result, 

\[
\langle V \rangle = \frac{\hbar \omega}{4} (n + n + 1) = \frac{1}{2}\hbar\omega
(n + \frac{1}{2}). 
\] \vspace{3px}

We see that the expectation value of the potential energy $V(x)$ is exactly
\textit{half} of the total energy $E = \left(n+\frac{1}{2}\right)\hbar\omega$.
This is a beautiful fact of the harmonic oscillator and a reason as to why it
shows up literally everywhere in physics. 

\subsection{Power Series Analytic Method}

I will now show the second, `brute-force' power series method to solve the
Schr\"odinger equation for the harmonic oscillator, 

\begin{align} \label{1Dharmonic}
  -\frac{\hbar^2}{2m} \frac{d^2 \psi}{d x^2} + \frac{1}{2}m\omega^2x^2\psi
  = E\psi
\end{align}\vspace{3px}

and solve it directly. We first introduce a dimensionless variable $\xi$ to
make things cleaner, where 

\begin{align} \label{xi}
\xi \equiv \sqrt{\frac{m\omega}{\hbar}}x
\end{align} \vspace{3px}

Hence, the Schr\"odinger equation now reads 

\begin{align} \label{1dharmonic}
  \frac{d^2 \psi}{d \xi^2} = (\xi^2 - K)\psi
\end{align}\vspace{3px}   

where $K \equiv \frac{2E}{\hbar\omega}$. Our problem is to solve Equation
\ref{1dharmonic} and in the process obtain the allowed values of $K$, and thus
$E$. To begin with, we notice that at very large $\xi$, (very large $x$ ),
$\xi^2$ completely dominates over the constant $K$, so in this circumstance, 

\begin{align} \label{approx_1}
  \frac{d^2 \psi}{d \xi^2} \approx \xi^2\psi
\end{align}

which has the approximate solution (verify) 

\begin{align} \label{}
  \psi(\xi) \approx Ae^{-\xi^2 / 2} + Be^{\xi^2 / 2}
\end{align}\vspace{3px}

And since the $B$ term blows up as $|x| \rightarrow \infty$, we get rid of it
to ensure $\psi(\xi)$ is normalizable. The physically acceptable solutions are
then 

\begin{align} \label{large xi}
  \psi(\xi) \rightarrow h(\xi)e^{\xi^2 / 2}, \quad \text{ at large $\xi$} 
\end{align}

where we replaced the constant $A$ with another function $h(\xi)$ in hopes that
it has a simpler functional form that $\psi(xi)$ itself. Differentiating
Equation \ref{large xi}, 


\begin{align} \label{}
  &\frac{d \psi}{d \xi} = \left( \frac{d h}{d \xi} -\xi h \right) e^{\xi^2 / 2}
  \\
  &\frac{d^2 \psi}{d \xi^2} = \left( \frac{d^2 h}{d \xi^2} - 2\xi \frac{d h}{d
  \xi}  + (\xi^2 - 1)h \right) e^{-\xi^2 / 2} \label{eq_2scr}
\end{align}\vspace{3px}

And therefore, equating Equation \ref{1dharmonic} and Equation \ref{eq_2scr}
transforms the Schr\"odinger equation into 

\begin{align} \label{1dscrf}
  \frac{d^2 h}{d \xi^2} - 2\xi \frac{d h}{d \xi}  + (K-1)h = 0
\end{align}\vspace{3px}

And here is where we utilize power series. We search for solutions to Equation
\ref{1dscrf} in the form of a power series in $\xi$. 

\begin{align} \label{}
  h(\xi) = a_0 + a_1\xi + a_2\xi^2 + \cdots = \sum_{j=0}^\infty a_j \xi^j
\end{align}\vspace{3px}

Differentiating the power series term by term to determine $ \frac{d h}{d \xi}
$ and $ \frac{d^2 h}{d \xi^2} $, 

\begin{align} \label{xidiffeqs}
  \frac{d h}{d \xi} &= a_1 + 2a_2\xi + 3a_3\xi^2 + &&\cdots = \sum_{j=0} ja_j
  \xi^{j-1} \\ 
  \frac{d^2 h}{d \xi^2} &= 2a_2 + 2 \cdot 3a_3\xi + 3 \cdot 4a_4\xi^2 + &&\cdots
  = \sum_{j=0}^\infty (j+1)(j+2)a_{j+2}\xi^j
\end{align}\vspace{3px}

Putting all these equations into Equation \ref{1dscrf}, we find

\begin{align} \label{ps}
  \sum_{j=0}^\infty [(j+1)(j+2)a_{j+2} - 2ja_j + (K-1)a_j]\xi^j =  0
\end{align}\vspace{3px}

It follows that the coefficient of \textit{each power} of $\xi$ must vanish
(since the whole thing equals 0). In other words

\[
  (j+1)(j+2) a_{j+2} - 2ja_j + (K-1)a_j = 0
\] \vspace{3px}

Rearranging gives us a recurrence relation, 

\begin{align} \label{rec}
  a_{j+2} = \frac{(2j+1-K)}{(j+1)(j+2)}a_j
\end{align}\vspace{3px}

Starting with $a_0$, we can generate all the even-numbered coefficients.
Starting with $a_1$, we can generate all the odd-numbered coefficients.
Therefore, we can write $h(\xi)$ as 

\begin{align} \label{}
  h(\xi) = h_\text{even} (\xi) + h_\text{odd} (\xi)
\end{align}\vspace{3px}

where 

\begin{align} \label{}
  h_\text{even} &= a_0 + a_2\xi^2 + a_4\xi^4 \\ 
  h_\text{odd} &= a_1\xi + a_3\xi^3 + a_5\xi^5
\end{align}\vspace{3px}

For very large $j$, the recursion formula becomes approximately, 

\[
  a_{j+2} \approx \frac{2}{j}a_j
\] \vspace{3px}

with the solution 

\[
a_j \approx \frac{C}{(j/2)!}
\] \vspace{3px}

for some constant  $C$. This yields (for very large $\xi$ ), 

\[
  h(\xi) \approx C\sum \frac{1}{(j/2)!}\xi^j \approx C\sum \frac{1}{j!}\xi^{2j}
  \approx Ce^{\xi^2}
\] \vspace{3px}

However, something seems wrong. If $h \sim e^{+\xi^2}$, then $\psi(\xi)
= h(\xi)e^{-\xi^2/2} \sim e^{+\xi^2/2}$ -- which blows up for large $\xi$, (and
hence for large $x$ ). There is only one way to resolve this. For normalizable
solutions, \textbf{the series must terminate at some j}. There must occur some
``highest $j$ ". (call it $j_\text{max} $, such that the recursion formula
spits out $a_{j_\text{max}  + 2} = 0$, which will make every coefficient
afterwards also 0. Therefore, for physically acceptable solutions, Equation
\ref{rec} requires 

\[
K = 2j_\text{max} + 1
\] \vspace{3px}

for some positive integer $j_\text{max} $. And since $K
= \frac{2E}{\hbar\omega}$, we recover the energy equation for a harmonic
oscillator 

\begin{align} \label{}
  E_j = \left( j + \frac{1}{2} \right) \hbar\omega, \quad \text{ for } j = 0,
  1, 2, \hdots  
\end{align}\vspace{3px}

We recover, by a completely different method, the fundamental quantization
condition we algebraically uncovered in Equation \ref{harmenergy}. 

For the allowed values of $K$, the recursion formula reads

\begin{align} \label{allowedrec}
  a_{j+2} = \frac{-2(j_\text{max} - j)}{(j+1)(j+2)}a_j
\end{align}\vspace{3px} 

If $j_\text{max} $ = 0, there exists only one term in the series, $h_0(\xi)
= a_0$. And hence 

\[
  \psi_0(\xi) = a_0e^{-\xi^2 / 2} \qquad \text{ if $j_\text{max} = 0 $ }
\] \vspace{3px}

which, apart from normalization is equivalent to Equation \ref{lowestrung}. For
$j_\text{max} =1$, we take $a_0 = 0$, and Equation \ref{allowedrec} with $j=1$ yields $a_3
= 0$, so 

\begin{align} \label{}
  h_1(\xi) = a_1\xi \quad \Rightarrow \quad \psi_1(\xi) a_1\xi e^{-\xi^2 / 2}
\end{align}\vspace{3px}

For $j_\text{max} = 2$, $j=0$ yields $a_2 = -2a_0$, and $j=2$ gives $a_4 = 0$,
so 

\begin{align} \label{}
  h_2(\xi) = a_0 \left( 1 - 2\xi^2 \right) \quad \Rightarrow \quad \psi_2(\xi)
  = a_0 (1-2\xi^2) e^{-\xi^2 / 2}
\end{align}\vspace{3px}   

and so on. The polynomials for $h(\xi)$ that are generated are known as
\textbf{Hermite Polynomials}. There are actually two different forms of Hermite
Polynomials -- the ``physicist's" and the ``probabilist's," that are related.
Obviously, the type us superior physicists use are the physicist's Hermite
Polynomials. By convention, the arbitrary multiplicative factor is chosen so
that the coefficient of the highest power of $\xi$ is $2^{j_\text{max} }$.
Then, the normalized stationary states for the harmonic oscillator are 

\begin{align} \label{powerseriessoln}
  \psi_n(x) = \left( \frac{m\omega}{\pi \hbar}\right)^{1/4} \frac{1}{\sqrt{2^n
  n!}} H_n(\xi) e^{-\xi^2 / 2}
\end{align}\vspace{3px}

where we use $j_\text{max} = n$ to label the quantum number of each state. These are identicial of course to the ones we obtained algebraically in
Equation \ref{laddersolution}. I plot the solutions $\psi_n(x)$ in Figure
\ref{graphsoln} below. 

\begin{figure}[!ht]
  \centering
    \includegraphics[width = 11cm, height = 8cm]{harmonicoscillatorsoln.pdf}
    \caption{Stationary State Wave Functions for 1D Harmonic Oscillator}
    \label{graphsoln}
\end{figure}

Note the energies are evenly spaced as $E_n = \left( n + \frac{1}{2} \right)
\hbar\omega $. We can afterwards determine $\Psi(x, t)$ via the prime
directive using Equation \ref{primedirec}. 

\subsection{The Free Particle}

We have previously (in studying the infinite square well) considered a particle
that could propagate freely in the region $-a/2 < x < a/2$, but was confined at
the boundaries by an infinitely strong potential. The solutions of the
time-independent Schr\"odinger equation were states of definite energy. We
described these states as $\sin$ and $\cos$ functions -- standing waves with
fixed nodes, formed by combining left-moving with right-moving amplitudes.
These states were normalizable and provided a complete orthonormal basis for
describing the propagation of any wave packet $\Psi(x, t)$. 

Here we consider a similarly free propagating particle, but one not confined by
any potential. Instead the particle is free -- able to move over the region
$-\infty < x < \infty$. Here however, we will choose to describe the waves as
plane waves in the form $e^{ikx}$ rather than with $\sin$ and $\cos$. These
states, as you shall soon see, are not normalizable and are thus not true
stationary states, yet they are still of use as they form a basis for expanding
physical states -- wave packets -- that \textit{are} normalizable physical
states. 

As the particle is free $(V(x) = 0)$ for all $-\infty < x < \infty$, the
Schr\"odinger equation for the time-independent stationary states is the
following: 

\begin{align} \label{}
  -\frac{\hbar^2}{2m} \frac{d^2 \psi(x)}{d x^2} = E\psi(x)
\end{align}\vspace{3px}

In terms of the wave number 

\[
k = \frac{\sqrt{2mE}}{\hbar} \quad \Rightarrow \quad \frac{d^2 \psi(x)}{d x^2}
= -k^2\psi(x)
\] \vspace{3px}

And therefore the general time-independent solution is 

\begin{align} \label{}
  \psi(x) = Ae^{ikx} + Be^{-ikx}
\end{align}\vspace{3px}

And the full time-dependent solution is therefore, via the prime directive, 

\begin{align} \label{}
  \Psi(x, t) = Ae^{ikx - iEt/\hbar} + Be^{-ikx - iEt/\hbar} \quad k \text{
  positive} 
\end{align}\vspace{3px}


If we allow $k$ to run over both positive and negative values, then $k = \pm
\frac{\sqrt{2mE}}{\hbar}$, and this simplifies the time-dependent solution into 


\begin{align} \label{}
  \Psi(x, t) = Ae^{ikx - iEt/\hbar} = Ae^{i\left( kx - \frac{\hbar k^2}{2m}t
  \right) } \quad k \text{ positive or negative}   
\end{align}\vspace{3px}


We can identify the \textit{velocity} of our solutions by jumping on the wave
function -- hanging onto a point of fixed phase -- and measuring which way we
travel. We take a positive step in time $\Delta t$ and demand that the phase
remain fixed 

\begin{align} \label{}
  kx - \frac{\hbar k^2}{2m}t \rightarrow k(x + \Delta x) &- \frac{\hbar
  k^2}{2m}(t + \Delta t) \Rightarrow k\Delta x - \frac{\hbar k^2}{2m}\Delta
    t = 0 \\ &\Rightarrow \frac{\Delta x}{\Delta t} \equiv v = \frac{\hbar
    k }{2m} \label{quantumv}
\end{align}\vspace{3px}

Therefore, our solutions with positive $k$ have a positive velocity (traveling
to the right) while those with negative $k$ have negative velocity (traveling
to the left). 

\begin{alignat*}{3}
  k &= +\frac{\sqrt{2mE}}{\hbar} > 0 \rightarrow \text{ traveling to the right }&& (+x)  
    \\ k &= -\frac{\sqrt{2mE}}{\hbar} < 0 \rightarrow \text{ traveling to the left } && (-x)
  \end{alignat*}

We can define the wavelength as a positive number 

\begin{align} \label{wavelengthk}
  \lambda = \frac{2\pi}{|k|}
\end{align}\vspace{3px}

but include a sign in the de Broglie relationship for momentum, 

\[
 p = \frac{2\pi\hbar}{\lambda}\rightarrow (2\pi\hbar) \left( \frac{k}{2\pi}
 \right) = \hbar k 
\] \vspace{3px}

so that momentum becomes a signed quantity (positive for waves moving to the
right, negative for those moving to the left). 

However, there is are two issues with our calculations. The first is regarding
our calculated wave velocity. The velocity of the
waves we calculated in Equation \ref{quantumv} is 

\begin{align} \label{}
  v_\text{quantum} = \frac{\hbar |k|}{2m} = \sqrt{\frac{E}{2m}}
\end{align}\vspace{3px}


On the other hand, the \textit{classical} speed of a free particle with energy
$E$ is given by $E = \frac{1}{2}mv^2$, so 

\begin{align} \label{}
  v_\text{classical} = \sqrt{\frac{2E}{m}} = 2v_\text{quantum} 
\end{align}\vspace{3px}

We get a quantum mechanical wave function that travels at \textit{half} the
speed of what the particle \textit{should}. The second issue is the fact that
our calculated wave function \textit{is not normalizable}. 

\begin{align} \label{}
  \int_{-\infty}^{\infty}  |\psi(x, t)|^2 \, dx = |A|^2 \int_{-\infty}^{\infty}
   \left( e^{-ikx + iEt/\hbar} \right) \left(
  e^{ikx - iEt/\hbar} \right)  \, dx \quad \rightarrow \quad |A|^2 (\infty)   
\end{align}\vspace{3px}


Therefore, our solutions do not represent physically realizable states. A free
particle can not exist in a stationary state -- \textit{ there is no such thing
as a free particle of definite energy.} But like I said previously, they still
serve purpose. They play a \textit{mathematical} role entirely dependent of
their \textit{physical} interpretation. The general solution to the
time-dependent Schr\"odinger equation via the prime directive is still a linear
combination of our stationary states. Only this time its an \textit{integral}
over the continuous variable $k$ instead of a discrete sum over index $n$. 

\begin{align} \label{freeparticlesolution}
  \Psi(x, t) = \frac{1}{\sqrt{2\pi}}\int_{-\infty}^{\infty} \phi(k) e^{i\left(
  kx - \frac{\hbar k^2}{2m}\right) } \, dk 
\end{align}\vspace{3px}

The quantity $\frac{1}{\sqrt{2\pi}}$ is factored out for convenience.e This
wave function \textit{can} be normalized. But it carries a \textit{range} of
$k$s, and hence a range of energies and velocities. This is what we call
a \textbf{wave packet}.  

In a normal quantum mechanics problem, we are provided $\Psi(x, 0)$ and asked
to find $\Psi(x, t)$. For a free particle the solution takes the form of
Equation \ref{freeparticlesolution}. But how do we determine $\phi(k)$? so as
to match the initial wave function

\[
  \Psi(x, 0) = \frac{1}{\sqrt{2\pi}} \int_{-\infty}^{\infty} \phi(k)e^{ikx} \,
  dk
\] \vspace{3px}

The answer, as you may have noticed given the form of Equation
\ref{freeparticlesolution} is via Fourier/Inverse fourier Transform. 

\[\boxed{
  f(x) = \frac{1}{\sqrt{2\pi}}\int_{-\infty}^{\infty} F(k)e^{ikx} \, dk \quad
  \Longleftrightarrow \quad F(k) = \frac{1}{\sqrt{2\pi}} \int_{-\infty}^{\infty}
f(x) e^{-ikx} \, dx}
\] \vspace{3px}


$F(k)$ is the \textit{fourier transform} of $f(x)$, and likewise $f(x)$ is the
\textit{inverse fourier transform} of $F(k)$. These integrals exist if the
initial wave packet $\Psi(x, 0)$ is normalized. The solution for a free
particle is Equation \ref{freeparticlesolution} with 

\begin{align} \label{solvingphi}
  \phi(k) = \frac{1}{\sqrt{2\pi}} \int_{-\infty}^{\infty} \Psi(x, 0)e^{-ikx} \,
  dx
\end{align}\vspace{3px}

Putting everything together, 

\begin{mainbox}{Wave Packet $\rightarrow$ Wave Function}
  Given an arbitrary initial wave packet $\Psi(x, 0)$. \\

  For a discrete system (Harmonic Oscillator, Infinite Square Well,  $\hdots$)

   \[
  \Psi(x, 0) = \sum_n c_n \psi_n(x) \qquad c_n = \int_\Omega \psi_n^*(x)\Psi(x,
  0) \, dx
  \] \[ \Psi(x, t) = \sum_n c_n \psi_n(x) e^{-iE_n t/\hbar} \] \vspace{3px}
  
where $\psi_n(x)$ and $E_n$ are the solutions to the discrete problem, and
where the integration in $x$ is over the region $\Omega$ where $\psi_n$ is
nonzero. \\

For a free particle, with its continuous plane-wave basis, 

\[ \Psi(x, 0) = \int_{-\infty}^{\infty} \phi(k)\psi(k, x) \, dk \qquad \phi(k)
= \int_{-\infty}^{\infty} \psi^*(k, x)\Psi(x, 0) \, dx \] \[ \Psi(x, t)
= \int_{-\infty}^{\infty} \phi(k)\psi(k, x) e^{-iE(k) t /\hbar} \, dk \]
\vspace{3px}
  
where $\psi(k, x) \equiv \sqrt{\frac{1}{2\pi}} e^{ikx}$ and $E(k)
= \frac{\hbar^2 k^2}{2m}$.
\end{mainbox}


\subsubsection{Worked Example} 

We start with a normalized wave packet that has been confined to a region of
width $2a$, 

\begin{align} \label{problemfree}
\Psi(x, 0) = \begin{cases}
  \frac{1}{\sqrt{2a}} &\qquad -a < x < a \\ 0 &\qquad |x| > a
\end{cases}
\end{align} \vspace{3px}

Using Equation \ref{solvingphi}, 

\[
  \phi(k) = \int_{-\infty}^{\infty} \left( \frac{1}{\sqrt{2\pi}}e^{-ikx}
  \right) \Psi(x, 0) \, dx = \frac{1}{\sqrt{2\pi}}\frac{1}{\sqrt{2a}}
  \int_{-a}^{a} \cos kx \, dx = \frac{1}{\sqrt{\pi a}} \frac{\sin ak}{k}
\] \vspace{3px}

Then finally, via the prime directive, or by using Equation
\ref{freeparticlesolution}, 

\begin{align} \label{solutionfree}
  \Psi(x, t) = \int_{-\infty}^{\infty} \phi(k)\psi(k, x) e^{-iE(k)t/\hbar} \,
  dk = \int_{-\infty}^{\infty} \frac{1}{\sqrt{\pi a}} \frac{\sin ak}{k} \left(
  \frac{1}{\sqrt{2\pi}} e^{ikx}\right)  e^{-i\hbar k^2 t / 2m} \, dk
\end{align} \vspace{3px}

Below I plot the initial probability density function at $|\Psi(x, 0)|^2$
along with the probability density function $|\Psi(x, ma^2 / \hbar)|^2$, at
a later time $t = ma^2 / \hbar$ (the curve) in Figure \ref{freegraph}. As you
can see, the probability density function and the wave function itself begin
to delocalize over time.

\begin{comment}
\begin{figure}[!ht]
  \centering
    \includegraphics[width = 12cm]{wavepacket.pdf}
    \caption{Plot of $|\Psi(x, 0)|^2$ and $|\Psi(x, ma^2 / \hbar)|^2$ for the
      wave function defined in Equation \ref{solutionfree} derived from the
      initial wave function in Equation \ref{problemfree}.}
\end{figure}
\end{comment}

\begin{figure}[H]
  \centering
    \includegraphics[width = 16cm]{wavepacket.pdf}
    \caption{Plot of $|\Psi(x, 0)|^2$ and $|\Psi(x, ma^2 / \hbar)|^2$.}
    \label{freegraph}
\end{figure}









