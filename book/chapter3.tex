\section{The Classical Wave Equation}

\begin{figure}[H]
\begin{subfigure}[b]{.59\linewidth}
\centering
\includegraphics[width=\linewidth]{damped_wave.pdf}
\caption{$A_0 \frac{\sin r}{r}$}\label{fig1a}
\end{subfigure}\hfill
\begin{subfigure}[b]{.39\linewidth}
\centering
\includegraphics[width=\linewidth]{damped_wave_1d.pdf}
\caption{1D cut}\label{fig1b}
\end{subfigure}%
\caption{Example of a very intuitive wave phenomena, but mathematically
a slightly complicated one. The one-dimensional cut of this wave shows the
oscillation is damped.}
\end{figure}

Since the propagation of the wave is radial and the amplitude decreases as the
radius increases, the simplest stationary wave function reads: 

\[
  A(r) = A_0 \frac{\sin r}{r}
\] \vspace{3px}

where $A(r)$ is the displacement of the water at position $r$ and $A_0$
indicates the amplitude at $r=0$. A mathematically much simpler wave equation
is the sine wave displayed below: 


\begin{figure}[H]
  \centering
    \includegraphics[width = 10cm]{wave.pdf}
\end{figure}

The wave function of the sinusoidal wave above is given by

\begin{align} \label{eq:1}
A(x, t) = A_0 \sin (kx - \omega t)
\end{align}
\vspace{3px}

where $A_0 = 1$ and $k = 1$, $\omega t = 0$. Here $k$ is the wave number and
$\omega$ is the angular frequency. Note that the two parameters together
describe the \textit{phase velocity} of the wave: 

\begin{align} \label{eq:2}
  v = \frac{\omega}{k}
\end{align}\vspace{3px}

By taking the second derivative of Equation \ref{eq:1} with respect to $x$ and
i$t$ and using Equation \ref{eq:2}: 

\begin{align*}
  \frac{\partial^2 A(x, t)}{\partial x^2} &= -k^2 A_0 \sin(kx - \omega t)\\
  \frac{\partial^2 A(x, t)}{\partial t^2} &= -\omega^2 A_0 \sin(kx - \omega t)
\end{align*}

Therefore, 

\begin{align} \label{eq:3}
  \frac{\partial^2 A(x, t)}{\partial t^2} = \frac{\omega^2}{k^2}
  \frac{\partial^2 A(x, t)}{\partial x^2} = v^2 \frac{\partial^2
  A(x,t)}{\partial x^2} 
\end{align} \vspace{5px}

Equation \ref{eq:3} is the wave differential equation in classical mechanics.
All parameters in the equation are real and physical. For the three dimensional
problem this equation is given in the form of the Laplacian Operator
$\nabla^2$ :

\begin{align} \label{eq:4}
  \frac{\partial^2 A(\vec{r}, t)}{\partial t^2} = v^2\nabla^2 A(\vec{r}, t)
\end{align}
Note that, since the cosine function can also describe this wave, it is
possible to generalize this function via Euler's formula: 

\begin{align} \label{eq:5}
  A(x, t) = A_0 e^{i(kx - \omega t)}
\end{align} \vspace{5px}

\section{Schr\"odinger's Equation}

Now that we are equipped with the classical definition of a wave, we can imbue
Equation \ref{eq:5} with our knowledge of Energy and de Broglie's wavelength.
\\

We now know the energy of electromagnetic waves is carried out by quanta
(packets of energy) known as photons. Each photon carries with it an energy
given by Equation \ref{eq:energy}: 

\begin{align} \label{eq:photonenergy}
  E = h\nu = \hbar \omega \qquad \hbar = h / 2\pi
\end{align} 

We also know each photon is characterized by its de Broglie wavelength given by
Equation \ref{eq:deBroglie}

\begin{align} \label{eq:de Broglie wavelength}
  \lambda = \frac{h}{p}
\end{align}


By using Equation \ref{eq:de Broglie wavelength}, and the relation between the
wave number and wavelength $k = \frac{2\pi}{\lambda}$, we obtain 

\[
k = 2\pi \frac{p}{h} = \frac{p}{\hbar}
\] \vspace{3px}

And with the help of Equation \ref{eq:photonenergy}, we can recast the original
classical wave Equation \ref{eq:5} into the form 

\begin{align}\label{eq:psi}
  \Psi(x, t) = \Psi_0 e^{i(px - Et)/\hbar}
\end{align}

where we have replaced the displacement $A(x,t)$ with $\Psi(x, t)$ along with
the amplitude to indicate it is a quantum wave function. \\

Now, since we are talking about a non-relativistic free particle described by
a plane wave, the total energy of the particle is just its kinetic energy, $E
= p^2 / 2m$. By using this fact and taking the first derivative of Equation
\ref{eq:psi} with respect to $t$, as well as the second derivative with respect
to $x$, we obtain 

\begin{align} \label{eq:derivatives}
  i\hbar \frac{\partial \Psi(x, t)}{\partial t}  = -\frac{\hbar^2}{2m}
  \frac{\partial^2 \Psi(x, t)}{\partial x^2} 
\end{align}

which is the wave equation for a free particle of mass $m$. Since the right
hand side of Equation \ref{eq:derivatives} corresponds to the kinetic energy
$KE$, we may generalize this equation to describe a particle moving under the
influence of a potential energy $V$. In the three dimensional coordinate system
we get

\begin{align} \label{eq: 3D Schrodinger Equation}
  i\hbar = -\frac{\hbar^2}{2m} \nabla^2 \Psi(\vec{r}, t)
  + V(\vec{r})\Psi(\vec{r}, t)
\end{align}

where we have used the fact that the total energy $E = KE + V$. Equation
\ref{eq: 3D Schrodinger Equation} is the Schr\"odinger Equation in its general
form. In one dimension, Schr\"odinger's Equation takes the form 

\begin{align} \label{eq: 1D Schrodinger Equation}
  i\hbar \frac{\partial \Psi(x, t)}{\partial t} = \left[ -\frac{\hbar^2}{2m}
  \frac{\partial^2 }{\partial x^2} + V(x) \right] \Psi(x, t). 
\end{align}

This completes the derivation of the Schr\"odinger Equation. It is not rigorous
by any means, however I feel it provides an intuition behind Schr\"odinger's
Equation that many Quantum Mechanics textbooks/courses skim over.

\twocolumn
\section{The Correspondence Principle}

The first chapter stressed that conceptional progress in physics is usually
a process where an existing theory is not replaced by, but is instead subsumed
into a more general theory that extends the scope and range of validity of the
original theory. That is, Quantum Mechanics should not \textit{replace}
classical mechanics, but instead should include it within its wider scope.

This concept is incorporated in quantum mechanics via the
\textit{correspondence principle}. It states that classical mechanics emerges
as a limit of quantum mechanics for large quantum numbers. 
For example, the bound states of a hydrogen atom have energies
of -13.6 $\text{eV} /n^2$. Thus $n$ can be increased without limit, and this
process ultimately produces bound orbits with binding energies that approach
zero, and with radii that steadily increase, scaling as $n^2$. Hydrogen (and
other atoms) in such highly excited bound states are called $Rydberg atoms$
. Their electron orbits approach the classical limit -- the electron moves in
a Keplerian orbit. This can be shown explicitly. 

Another aspect of the correspondence principle is that the classical limit can
also be obtained by altering quantum mechanics by taking the limit $\hbar
\rightarrow  0$. In classical mechanics when one throws a ball from point $x$
at time $t_x$ that is caught by a receiver at point $y$ at time $t_y$, the ball
follows a precise path which we can calculate from Newton's laws. The path
minimizes the action -- the difference between the kinetic energy and potential
energy, integrated along the time coordinate of the classical path, 

\[
  S = \oint_{t_x}^{t_y} [KE - V]\, dt.
\] \vspace{3px}

If you sample any path other than the one given by Newton's Laws, the action
along the path will be higher. Note that the action carries the same units as
Planck's constant $h$ -- energy $\times$ time. 

There is formulation of quantum mechanics, equivalent to the one we will use in this class, in
terms of paths. It provides a very intuitive picture of the relationship of quantum mechanics and
classical mechanics. In quantum mechanics you are allowed to propagate from $x$
to $y$ by many paths -- but the further a path deviates from the classical path, the less probable it is. 
One pays a ``penalty" for increasing the action via a path other than the
classical path -- the bigger the increase in the action, the stiffer the
penalty. Now to convert action to a number -- something that could
possibly lead to a probability -- one needs a unit. $h$ is that unit. The classical path remains the
best path, but there are many many others that, though each may be somewhat less probable than
the classical path, will contribute. The larger the deviation of the action from its classical path,
the bigger the penalty one pays in units of $h$, and thus the less probable the path. So the difference
between classical mechanics and quantum mechanics is that the former has a single defined path,
while in the latter many, many paths are allowed, `fuzzying' (that’s probably not a word!) out
the classical path -- but not too much because of the heavy penalty one pays for taking a distant
path. Classical mechanics is recovered by taking $h \rightarrow 0$. All
penalties become infinite, so only the classical path is allowed.  This is
a beautiful way to think about Planck's constant, though is a little
`hand-wavy.'

The penalty one pays actually arises from the \textit{interference} among
paths. If we represent the action by $S$, the weighting of a path is 

\[
  e^{iS/\hbar}
\] \vspace{3px}
so that a large excess in the action causes rapid fluctuations with respect to the classical path and
its nearest neighbors, leading to destructive interference among paths. In
contrast, paths near the classical path have slowly varying relative phases, and thus tend to cohere. 
If one drives $\hbar \rightarrow 0$ constructive interference among paths only
occurs for paths increasingly near the classical one. At $\hbar = 0$ one converges to the classical path.
I think this gives one a much deeper feel for the physics of Planck’s constant
-- how it governs the deviations from classical mechanics – and helps one
visualize how the classical limit is achieved as $\hbar \rightarrow 0$. In
summary, 

\begin{mainbox}{Correspondence Principle}
  Quantum Mechanics becomes Classical as
  \[ n \rightarrow \infty \quad \text{or as} \quad \hbar \rightarrow 0 .\]
\end{mainbox}


\section{The Principle of Superposition \& Wave Packets}

Any theory that would generalize classical mechanics should be required the
reproduce classical mechanics in appropriate limits. The accumulation of
phenomena in the early years of the 20th century indicating that light could
behave as a particle and that particles could behave as waves, led us to a path
where particles with definite positions and momenta gave way to a description
in terms of waves and wave packets. Therefore, before we introduce the wave
equation, we should remind ourselves of some of the properties achievable
through waves. 

Familiar wave equations are those for sound in air or waves in water, which in
1D take the form 

\[
\frac{\partial^2 \psi}{\partial x^2} = \frac{1}{c^2} \frac{\partial^2
\psi}{\partial t^2} 
\] \vspace{3px}

We can look for solutions of this equation in the form of an oscillation wave. 

\[
  \psi(x, t) = \phi(x)e^{i\omega t} \quad \Rightarrow \quad \frac{\partial^2
  \phi(x)}{\partial x^2} = -\frac{\omega^2}{c^2}\phi(x)
\] \vspace{3px}

and we find 

\[
\phi(x) = e^{ikx} \quad \text{where} \quad c ^2k^2 = \omega^2 \] so
  that \[ \qquad \psi(x,t) = e^{i(kx - \omega t)} \quad \text{with} \quad  k(\omega)
    = \pm \frac{w}{c}
  \] \vspace{3px}

This is the solution we found for Equation \ref{eq:5}. These plane wave
solutions are extended, covering the entire range of $x$. An important property
of this equation is that it is linear in $\psi$. This leads to the principle of
superposition: 

\begin{subbox}{Principle of Superposition}
  If $\psi_1(\vec{x}, t)$ and $\psi_2(\vec{x}, t)$ satisfy the wave equation,
  so does $\psi_1(\vec{x}, t) + \psi_2(\vec{x}, t), t).$
\end{subbox}

This property allows one to build \textit{wave packets}, as you would get by
throwing a stone into the middle of a quiet pond. Such a localized wave can be
made from superpositions of the extended plan waves derived above. An example
is given below. 

It would be very difficult to envision a successful theory of Quantum Mechanics
that lacked this property. The correspondence principle requires us to be able
to create localized particles, and we know how to build localized wave packets
from waves via Fourier Analysis: 

\[
  \psi(x, t=0) = \frac{1}{\sqrt{2\pi}}\int_{-\infty}^{\infty} \phi(k)e^{ikx}\,
dk \] \[ \quad \phi(k) = \frac{1}{\sqrt{2\pi}}\int_{-\infty}^{\infty} \psi(x,
t=0)e^{-ikx}\, dx\] \vspace{3px}

For example, let

\[
\psi(x, t=0) = e^{-x^2 / a^2} \] Then, \[ \psi(x, t=0)
  = \frac{1}{\sqrt{2\pi}}\int_{-\infty}^{\infty} e^{-a^2k^2/4}e^{ikx} \, dk
\] \vspace{3px}

We can easily build a Gaussian wave packet out of plane waves, but the
equivalent infinite sum of plane waves would not be a solution of our quantum mechanical
wave equation unless the superposition principle holds. 

\subsection{Wave Packets \& Uncertainty Relationships}

The simple Gaussian example shown above illustrates another important property
of wave packets. There is a size scale associated with our coordinate-space
wave packet, with $\Delta x \sim a$. But we see the smaller $a$ -- the more
localized in $x$ -- the broader the range of contributing momentum-space ($k$)
waves. That is, 

\[
  e^{-a^2k^2/4} = e^{-k^2 / (2/a)^2} \quad \Rightarrow \quad \Delta k = \frac{2}{a}
\] \vspace{3px}

Consequently $\Delta x \Delta k \sim 1$. A property of wave packets is the more
they are \textit{localized} in coordinate ($x$) space, the more they
\textit{delocalize} in momentum space. Thus as a wave  theory, we would expect
quantum mechanics to have an \textit{uncertainty principle} that prevents us
from simultaneously having particle locations and momenta. As the de Broglie
relation (Equation \ref{eq:deBroglie}) gives us $p = \frac{h}{\lambda}
= \frac{h}{2\pi}k$ so that $\Delta k = \Delta p / \hbar$, it is not surprising
that Quantum Mechanics has an uncertainty principle relating the product of
coordinate and momentum uncertainties to $\hbar$. Its precise form is 

\[
\Delta x \Delta p \geq \frac{\hbar}{2}
\] \vspace{3px}

We will do a more precise and rigorous derivation of \textit{Heisenberg's Uncertainty
Principle} utilizing the Cauchy-Schwarz Inequality in the near future. 

\section{Back to Schr\"odinger} 

In 1D, Schr\"odinger's Equation takes the form 

\[
  \left[ -\frac{\hbar^2}{2m} \frac{\partial^2 }{\partial x^2} + V(x)\right]
  \Psi(x, t) = i\hbar \frac{\partial }{\partial t} \Psi(x, t)
\] \vspace{3px}

with the external potential $V(x)$ unspecified. As we saw, this equation is
a fairly gentle variation of the sound/water wave equation we discussed, with
a couple interesting differences. 

So what does this equation mean? Staring at the LHS, the potential is clear,
while the derivative term can be rewritten in a way that clarifies units

\[
-\frac{\hbar^2}{2m} \frac{\partial^2 }{\partial x^2}
= -\frac{\hbar^2c^2}{2mc^2} \frac{\partial^2 }{\partial x^2} 
\] \vspace{3px}

As  $(\hbar c)^2$ has units of (Energy-Distance)$^2$,  $mc^2$ is an energy, and
$ \frac{\partial^2 }{\partial x^2} $ has units of (distance)$^{-2}$, the first
term on the LHS is an energy, and by the ``what else can it be" argument, must
be the kinetic energy. Classically this is $p^2 / 2m$. But $p^2/2m + V = E$, so
on the RHS, the differential operator must be generating $E$. The requirement
that Schr\"odinger's Equation with its differential operator corresponds with
energy conservation allows us to identify the equations differential operators
with a classical view. 

We define the momentum operator $\hat{p}$ 

\begin{align}\label{eq:momentum}
  \hat{p} \equiv \frac{\hbar}{i} \frac{\partial }{\partial x} = -i\hbar
  \frac{\partial }{\partial x} 
\end{align}\vspace{3px}

And the energy operator $\hat{E}$

\begin{align}
  \hat{E} \equiv i\hbar \frac{\partial }{\partial t} 
\end{align} \vspace{3px}

Therefore, we can rewrite Schr\"odinger's Equation using these operators as
follows: 

\begin{align*}
  \left[ -\frac{\hbar^2}{2m} \frac{\partial^2 }{\partial x^2} +V(x)\right]
  \Psi(x, t) &= i\hbar \frac{\partial }{\partial t} \Psi(x, t) \\ 
  \left[ \frac{\hat{p}^2}{2m} +V(x)\right] \Psi(x, t) &= \hat{E}\Psi(x, t)
\end{align*}

We indicate that $\hat{p}$ and $\hat{E}$ are Quantum Mechanical operators by
giving them ``hats." They are differential operators that act on the wave
function. 

One of the most important differences between the QM wave equation and the wave
equation for sound/water, is that the latter is quadratic in its space and time
differential operators, while the Schr\"odinger Equation is quadratic in space
but \textit{linear in time}. The time-linearity of the Schr\"odinger Equation
is a reflection of the non-relativistic relationship between a particles
momentum and its energy -- a consequence of building theory beyond classical
mechanics that extends to our reach and beyond the atomic scale, but shares
with the nonrelativistic restrictions associated with classical mechanics. The
linear-in-time nature of the Schr\"odinger Equation naturally leads to complex
wave functions. As measurements involve \textit{real} quantities, the
connection between wave functions and observables requires discussion -- the
topic of next chapter. 

I will quickly also note that our plane-wave solution of the Schr\"odinger
Equation has a characteristic wavelength. The length of a wave corresponds to
the distance required to change the phase by $2\pi$, at a fixed time $t$. That
is, 

\[
2\pi = \frac{p\Delta x}{\hbar} \equiv \frac{p\lambda}{\hbar} 
\] \vspace{3px}
Thus, 

\[
\lambda = \frac{2\pi\hbar}{p} = \frac{h}{p}
\] \vspace{3px}

That is precisely de Broglie's wavelength! 

\onecolumn

